\chapter{Conclusion and Future Work}
\label{ch:concl}

\section{Conclusion}
In this project, we proposed an analytical landscape to deal with AIS data in Singapore Straits. The work in this project can be split into two parts; first, AIS descriptive-analytic. It includes understanding the underlying of AIS end-to-end communication system, discerning AIS features, and uncovering insight or pattern within the data. The second part is the AIS predictive analytic. The purpose of this part is to develop a predictive model for the vessel future movement. The main challenge of this work is how to give a foundation about AIS analytical discourse in Singapore region, from descriptive-analytic to predictive one. The main contribution of this work is to provide a method, from the methodology, technical library, and future direction, to help researcher investigate deeper into AIS analytic specifically in the Singapore region.

With the current position as the leading maritime capital in the world, Singapore is facing the need to enhance maritime security at Singapore water more than ever. The Singapore Straits is passed by about 100000 vessels each year, and an average of 1 million AIS message is broadcast per day, leading to about 800 messages per minute. AIS message contains valuable information for coastal surveillance, whereby the activity range from vessel monitoring, tracking or directing. The vessel's trajectory prediction model is developed to help monitor maritime activity and hopefully assist the coastal authority for a prompt emergency reaction. AIS exploratory data analysis is discussed beforehand to develop a better understanding of what AIS data can give insight about. The data exploration is accompanied by the statistical description as well as the visualization framework using several built-in tools and libraries.

Trajectory prediction model aims to give the user the future movement of a vessel for a certain timestep given its AIS historical information. We build a data processing pipeline that performs data cleaning, data interpolation, and data extraction. We also design and build two different data input representation for general Machine Learning and Recurrent Neural Network (RNN) model. An archetype of RNN called Long Short Term Memory (LSTM) has an overall better performance metrics as compared to a baseline Linear Regression model. We experiment with 2 interpolation techniques, 3 performance metrics, 3 prediction timestep and several window size. The purpose of the experiment is to analyze under which condition the LSTM model would perform well, the model limitation and the workaround. Prediction model using 1 timestep have high accuracy indicated by a low error performance metrics of MAE, RMSE, and MAPE. However single-step prediction is not useful in practice. Prediction model using 10 timesteps come into rescue with a good balance between prediction accuracy and longer timestep. In this experiment we discovered that MAPE as an error measure have a pitfall causing them to be unreliably high for certain time-series dataset despite the counterpart metric of MAE is low. Prediction model using 30 timesteps is considered inaccurate across all window size, interpolation technique, and most importantly all error performance metrics. One alternative solution without changing the model complexity for this problem is to utilize the 1 and 10 timestep prediction model to predict 30 timestep multiple times using the recursive multi-step technique. The technique is proven to be useful in predicting 30 timesteps without hurting the performance. The best prediction accuracy (lower error) is when the 10 timestep model used to generate 30 timestep prediction. The resulting error metrics go significantly lower from 0.423 MAE to 0.0628 MAE.

\section{Future Research Directions}
As for future work, we would like to leverage big data worth of 1 year of AIS data and use distributed computing framework for data processing in the hope of getting better insight out of AIS data. Another work improvement would be to improve the prediction model by experimenting with more complex architecture and selecting more features combining numerical and categorical data.

We believe the above (but not limited to) future research directions will advance the technology presented in this thesis and contribute to academia and industry.
